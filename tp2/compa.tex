
\section{Comparación}

% Breve comparación de “programming in the small" vs "programming in the large” (diseño OO vs Arquitectura). Conclusiones que sacó el grupo.

\textit{Programming-in-the-small} es un método de desarrollo usado para construir proyectos chicos de software, en los que normalmente trabajan pocas personas, quienes están en general informados acerca de todos los sectores del proyecto. Es habitual que no haya roles designados, y que cada uno de los participantes ``haga un poco de todo''. En ese sentido, es lo más común que el mismo equipo se dedique al diseño del software y después a escribir el código fuente basándose en su propio diseño.

En \textit{programming-in-the-large} existe necesariamente una gran cantidad de desarrolladores, habitualmente dividido entre varios equipos que a veces trabajan en distintas empresas, y con el agregado de componentes \textit{off-the-shelf} cuya estructura interna puede no ser conocida, y no está pensada para ser particularmente compatible con el resto del sistema  en el que se está trabajando. También es normal que exista un equipo especializado que se dedique a producir la arquitectura del sistema, para luego tener un equipo diferente encargado de desarrollar el código fuente. 

La especificación dada por la arquitectura está escrita a un nivel mucho más global que los documentos de diseño: un diseño orientado a objetos, por ejemplo, tiene un nivel de detalle en el que normalmente se especifica cada mensaje individual, y para los que son claves dentro de la funcionalidad se detalla casi cada línea de código.

En un proyecto grande también es muy probable que el software deba ejecutarse en distintas arquitecturas de hardware; muchos de estos productos manejan grandes volúmenes de datos, y suelen requerir estructuras de hardware especiales para mantenerlos. En un proyecto chico generalmente este tipo de preocupaciones no están presentes: el software suele ser una pieza que corre en un único dispositivo, y almacena sus datos en la memoria que este le provee; no se necesita nada accesorio.

\nota{FALTA MÁS DESARROLLO, COLABOREN :) }