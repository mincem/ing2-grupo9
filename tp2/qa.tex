%%%% ATRIBUTO DE CALIDAD %%%%

\newcommand{\QA}[8] {
\item
  Atributo de \textbf{#1}:\\\textit{``#2''}

  \begin{itemize}
  \item
    Fuente: #3
  \item
    Estímulo: #4
  \item
    Artefacto: #5
  \item
    Entorno: #6
  \item
    Respuesta: #7
  \item
    Medida: #8
  \end{itemize}
}

%%%%%%%%%%%%%%%%%%%%%%%%%%%%%

\section{Atributos de calidad}

% Descripción de atributos de calidad identificados, a través de escenarios, incluyendo prioridades relativas.

% %%%% PLANTILLA PARA ESCENARIO %%%%
% 
% \QA
%   {} % tipo de atributo
%   {} % texto del atributo
%   {} % fuente
%   {} % estímulo
%   {} % artefacto
%   {} % entorno
%   {} % respuesta
%   {} % medida de la respuesta
% 
% %%%%%%%%%%%%%%%%%%%%%%%%%%%%%%%%%%


  \begin{enumerate}

\QA
  {Modificabilidad} % tipo de atributo
  {El proyecto obtendrá datos de la actividad online de los usuarios, incluyendo varias redes sociales y comentarios en sitios de espectáculos. Se espera que periódicamente se agregue compatibilidad con nuevos sitios de Internet, sin perjudicar el uso de la aplicación.} % texto del atributo
  {Equipo de desarrollo} % fuente
  {Introduce un componente que obtiene datos de un determinado sitio de Internet y los procesa en el formato usado por el sistema} % estímulo
  {Sistema} % artefacto
  {Tiempo de ejecución} % entorno
  {El componente es agregado y empieza a funcionar, mientras el sistema continúa proveyendo los servicios normales} % respuesta
  {La incorporación del nuevo componente finaliza en menos de 20 horas de trabajo} % medida de la respuesta

\QA
  {Modificabilidad} % tipo de atributo
  {Es cada vez más importante el público multilingüe, por lo que debemos adaptarnos a la posibilidad de que mensajes en distintos idiomas se refieran a una misma emisión de un programa.} % texto del atributo
  {Administrador del sistema} % fuente
  {Quiere agregar soporte para un programa de TV en un nuevo idioma} % estímulo
  {Interfaz} % artefacto
  {En tiempo de ejecución} % entorno
  {Se realiza el cambio sin afectar el resto de la aplicación} % respuesta
  {Sólo se modifican los componentes correspondientes a los programas de TV} % medida de la respuesta

\QA
  {Disponibilidad} % tipo de atributo
  {Es fundamental garantizar que todos los nodos estarán siempre listos para procesar información a todo momento.} % texto del atributo
  {Interna del sistema} % fuente
  {Un nodo no responde a las consultas de los usuarios} % estímulo
  {Sistema} % artefacto
  {Operación normal} % entorno
  {Se reporta la falla y se solicita una reparación del nodo inhabilitado} % respuesta
  {El nodo vuelve a estar activo en el transcurso de una hora} % medida de la respuesta

\QA
  {Modificabilidad} % tipo de atributo
  {Las personas que mejoraron nuestro algoritmo de sentiment analysis quieren aprender y mejorar la performance del mismo de manera continua.} % texto del atributo
  {Equipo de desarrollo de Sentiment Analysis} % fuente
  {Implementan un módulo nuevo o mejorado para realizar S.A. y lo incorporan al producto} % estímulo
  {Sistema} % artefacto
  {Tiempo de ejecución} % entorno
  {El software se intercambia de forma transparente al usuario y el producto sigue funcionando} % respuesta
  {El cambio se realiza en menos de 8 horas de trabajo y no se deben modificar otros elementos del sistema} % medida de la respuesta

\QA
  {Usabilidad} % tipo de atributo
  {Se quiere tener una interfaz bonita y fácil de usar, para tener mayor aceptación entre el público.} % texto del atributo
  {Usuario de la aplicación} % fuente
  {Desea ver el rating actual de un programa de TV} % estímulo
  {Sistema} % artefacto
  {En tiempo de ejecución} % entorno
  {Se provee una interfaz de usuario fácil de entender y usar} % respuesta
  {El 90\% de los usuarios puede realizar la operación al usar por primera vez la aplicación, y demoran menos de 30 segundos} % medida de la respuesta

\QA
  {Disponibilidad} % tipo de atributo
  {El sistema siempre debe estar disponible, aunque uno de los nodos no esté funcionando correctamente.} % texto del atributo
  {Externa} % fuente
  {No hay respuesta del nodo más cercano} % estímulo
  {Canal de comunicación} % artefacto
  {Normal} % entorno
  {Se pide información al nodo replicador} % respuesta
  {Se obtiene respuesta con demora menor que el doble del tiempo de respuesta normal} % medida de la respuesta

\QA
  {Seguridad (Confidencialidad)} % tipo de atributo
  {Los nodos intercambiarán datos entre sí, y la información que se transmite entre los nodos debe ser segura y no poder alterarse desde afuera.} % texto del atributo
  {Atacante externo} % fuente
  {Captura un mensaje con información enviada entre nodos e intenta acceder a los datos confidenciales} % estímulo
  {Datos del sistema} % artefacto
  {Operación normal online} % entorno
  {El usuario no puede decodificar los datos} % respuesta
  {En el 99,9\% de los casos el atacante necesitaría más de 10 años para decodificar los datos} % medida de la respuesta

\QA
  {Disponibilidad} % tipo de atributo
  {La comunicación con el sistema de análisis de sentimiento SQPost suele fallar. Siempre que sea posible, se utilizará SQPost. En caso contrario, el sistema debe seguir respondiendo, y realizar el análisis con el sistema propio.} % texto del atributo
  {Interna del sistema} % fuente
  {Se recibe un timeout} % estímulo
  {Componente de SQPost} % artefacto
  {Normal} % entorno
  {Cambiar a modo de emergencia con análisis de sentimiento propio} % respuesta
  {El sistema no tarda más de 1 seg. en cambiar a modo de emergencia y continuar operando, desde que se descubrió la falla} % medida de la respuesta

\QA
  {Performance} % tipo de atributo
  {Comportamiento esperado en cuanto procesamiento de imágenes obtenidas por la cámara del SmartTv} % texto del atributo
  {Externa} % fuente
  {El SmartTv sacá fotografías que serán tenidas en cuenta para mejorar el análisis del rating} % estímulo
  {Sistema de procesamiento de imágenes} % artefacto
  {Normal} % entorno
  {Se delega el procesamiento necesario al Hardware del SmartTv permitiendo una mayor descentralización y capacidad de computo} % respuesta
  {El sistema logra procesar para un determinado televidente una foto cada diez segundos} % medida de la respuesta

\QA
  {Performance} % tipo de atributo
  {El procesamiento de información deberá tardar menos de un segundo} % texto del atributo
  {Externa} % fuente
  {El cliente hace un pedido de Rating o Popularidad} % estímulo
  {Sistema} % artefacto
  {Normal} % entorno
  {El sistema procesa el pedido y entrega una respuesta} % respuesta
  {El 90 porciento de las veces el pedido es entregado en tiempo y forma} % medida de la respuesta

\QA
  {Performance} % tipo de atributo
  {El sistema responda efectivamente a 100? pedidos por segundo} % texto del atributo
  {Externa} % fuente
  {Varios clientes hacen pedidos en una ventana de 1 segundo} % estímulo
  {Sistema} % artefacto
  {Normal} % entorno
  {El sistema atiende los primeros pedidos de forma normal y pasa a modo sobrecargado, donde no responde ciertos pedidos hasta que se normalice la situación} % respuesta
  {El sistema responde 100 pedidos por segundo de forma normal} % medida de la respuesta



\end{enumerate}

