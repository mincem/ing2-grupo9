%%%% ATRIBUTO DE CALIDAD %%%%

\newcommand{\QA}[8] {
\item
  Atributo de \textbf{#1}:\\\textit{``#2''}

  \begin{itemize}
  \item
    Fuente: #3
  \item
    Estímulo: #4
  \item
    Artefacto: #5
  \item
    Entorno: #6
  \item
    Respuesta: #7
  \item
    Medida: #8
  \end{itemize}
}

%%%%%%%%%%%%%%%%%%%%%%%%%%%%%

\section{Atributos de calidad}

% Descripción de atributos de calidad identificados, a través de escenarios, incluyendo prioridades relativas.

% %%%% PLANTILLA PARA ESCENARIO %%%%
% 
% \QA
% 	{} % tipo de atributo
% 	{} % texto del atributo
% 	{} % fuente
% 	{} % estímulo
% 	{} % artefacto
% 	{} % entorno
% 	{} % respuesta
% 	{} % medida de la respuesta
% 
% %%%%%%%%%%%%%%%%%%%%%%%%%%%%%%%%%%


  \begin{enumerate}


\QA
{Modificabilidad} % tipo de atributo
{Es cada vez más importante el público multilingüe, por lo que debemos adaptarnos a la posibilidad de que mensajes en distintos idiomas se refieran a una misma emisión de un programa.} % texto del atributo
{Administrador del sistema} % fuente
{Quiere agregar soporte para un programa de TV en un nuevo idioma} % estímulo
{Interfaz} % artefacto
{En tiempo de ejecución} % entorno
{Se realiza el cambio sin afectar el resto de la aplicación} % respuesta
{Sólo se modifican los módulos correspondientes a los programas de TV} % medida de la respuesta

\QA
{Usabilidad} % tipo de atributo
{Se quiere tener una interfaz bonita y fácil de usar, para tener mayor aceptación entre el público.} % texto del atributo
{Usuario de la aplicación} % fuente
{Desea ver el rating actual de un programa de TV} % estímulo
{Sistema} % artefacto
{En tiempo de ejecución} % entorno
{Se provee una interfaz de usuario fácil de entender y usar} % respuesta
{El 90\% de los usuarios puede realizar la operación al usar por primera vez la aplicación, y demoran menos de 30 segundos} % medida de la respuesta

\QA
{Seguridad (Confidencialidad)} % tipo de atributo
{Los nodos intercambiarán datos entre sí, y la información que se transmite entre los nodos debe ser segura y no poder alterarse desde afuera.} % texto del atributo
{Atacante externo} % fuente
{Captura un mensaje con información enviada entre nodos e intenta acceder a los datos confidenciales} % estímulo
{Datos del sistema} % artefacto
{Operación normal online} % entorno
{El usuario no puede decodificar los datos} % respuesta
{En el 99,9\% de los casos el atacante necesitaría más de 10 años para decodificar los datos} % medida de la respuesta

\QA
{Disponibilidad} % tipo de atributo
{La comunicación con el sistema de análisis de sentimiento SQPost suele fallar. Siempre que sea posible, se utilizará SQPost. En caso contrario, el sistema debe seguir respondiendo, y realizar el análisis con el sistema propio.} % texto del atributo
{Interna del sistema} % fuente
{Se recibe un timeout} % estímulo
{Componente de SQPost} % artefacto
{Normal} % entorno
{Cambiar a modo de emergencia con análisis de sentimiento propio} % respuesta
{El sistema no tarda más de 1 seg. en cambiar a modo de emergencia y continuar operando, desde que se descubrió la falla} % medida de la respuesta

\QA
{Disponibilidad} % tipo de atributo
{El sistema siempre debe estar disponible, aunque uno de los nodos no esté funcionando correctamente.} % texto del atributo
{Externa} % fuente
{No hay respuesta del nodo más cercano} % estímulo
{Canal de comunicación} % artefacto
{Normal} % entorno
{Se pide información al nodo replicador} % respuesta
{Se obtiene respuesta con demora menor que el doble del tiempo de respuesta normal} % medida de la respuesta


\end{enumerate}

