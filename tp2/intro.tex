
\section{Introducción}

% Alguien va a tener que escribir algo acá. - M.I.

Este trabajo es una descripción de la arquitectura del proyecto \emph{Rating del Siglo XXI}. Se pretende modelar una aplicación que calcule el rating y la popularidad de un programa de televisión a través de la participación de los televidentes en las redes sociales y otros sitios online de espectáculos. 

La información obtenida de cada red social se almacena en el sistema en un formato interno al que le damos el nombre de \emph{Post}, que consiste en un mensaje, con autor y fecha de publicación, en el que se menciona un determinado programa de TV. Estos posts son clasificados por un algoritmo de Sentiment Analysis, que los clasifica como positivos, negativos o neutros.

El rating de un programa de TV se calcula a a partir de los posts obtenidos para el programa durante su horario de emisión, y de la actitud de estos mensajes respecto del programa. La popularidad se refiere a la participación de los usuarios en cualquier horario, durante un período determinado de tiempo.

Los usuarios del sistema podrán acceder a su funcionalidad a través de Internet. El procesamiento de toda la información se realiza en uno de los servidores centrales, llamados \emph{nodos}, que corresponden a regiones geográficas de América Latina. Los usuarios se conectarán con la aplicación para realizar consultas, y desde los nodos se producirá la respuesta, que se envía al cliente.

En una etapa más avanzada de desarrollo se agregará al sistema una funcionalidad para recibir información de Sentiment Analysis de fotos tomadas por las cámaras de Smart TVs, y usarla para complementar los valores obtenidos con la funcionalidad original.
