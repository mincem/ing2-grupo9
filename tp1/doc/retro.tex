\section{Retrospectiva} 

% Opiniones del proyecto después de terminar la iteración 1. Cómo seguiríamos, qué hicimos mal y deberíamos mejorar, cómo nos rindió el tiempo, etc. - M.I.

Durante el último mes trabajamos en la primera iteración del ciclo iterativo incremental de desarrollo de la aplicación, según la metodología ágil de \emph{SCRUM}. Analizando nuestro método de trabajo y los resultados obtenidos, llegamos a las siguientes conclusiones:

En primer lugar, nos resultó difícil adaptarnos a la metodología de trabajo ágil; hubo momentos de poca comunicación en el grupo, y en algunos casos no logramos hacer una distribución de tareas tal que cada miembro del grupo tuviera claro quién tenía asignada cada responsabilidad. Además no logramos organizar las tareas en un primer momento, por lo que una primera parte del \emph{sprint} no fue aprovechada. Combinado con algunas obligaciones académicas de los miembros del grupo, esto resultó en que la cantidad de trabajo que realizamos fue aumentando a medida que el final del sprint se acercaba, en vez de ser relativamente constante como hubiera sido ideal.

Finalmente esto llevó a que no pudiéramos completar el trabajo en el tiempo establecido, y debimos postergarlo hasta una fecha de entrega distinta.
\bigskip

Por otro lado destacamos que, cuando organizamos una división de tareas con reuniones o charlas periódicas para mantenernos informados, pudimos avanzar más rápido con las tareas, en especial el desarrollo del código fuente, cada uno trabajando en módulos encapsulados.

También llegamos a la conclusión de que es necesario tener claro el diseño de la aplicación para poder escribir el código fuente. En un principio teníamos algunos módulos del proyecto mejor diseñados que otros, y nos ocurrió que, en los que tenían un diseño más vago o poco claro, encontrábamos objetos que parecían estar aislados del modelo, sin usarse, o atribuciones que estaban repetidas entre varios objetos. En algunos casos fue necesario reunirse entre los miembros del grupo para desambiguar ese tipo de situaciones, lo que nos forzó a detallar más el diseño de esa parte para poder entender el funcionamiento que estábamos imaginando. Creemos que para el futuro es importante tener un diseño global hecho, definir las interfaces o uniones entre módulos que sean relativamente estables (poco volátiles) y basar las siguientes decisiones en ese diseño.
\bigskip

Evaluando nuestra aplicación entre los miembros del grupo y con nuestro tutor, notamos que, si bien seguimos en su mayoría el diseño que escribimos, la parte más ``externa'' del programa está muy atada a la interfaz gráfica que usamos. En otras palabras, la GUI se encarga de realizar tareas como crear los objetos esenciales y procesar las consultas. Esto nos dificultaría en el caso de tener que portar la funcionalidad a otro tipo de interfaz, ya que habría que reescribir parte del código que realiza las tareas. Teniendo en cuenta que este es un prototipo y que el desarrollo es iterativo, lo ideal sería reconocer el problema durante la siguiente iteración, y agregar nuevos objetos que sirvan como puente entre la funcionalidad y cualquier GUI que se use, de forma de desacoplar la interfaz del resto del programa.